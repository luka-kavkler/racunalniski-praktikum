\documentclass[12pt]{article}
\usepackage[slovene]{babel}
\usepackage{amsfonts,amssymb,amsmath,mathrsfs,amsthm}
\usepackage[utf8]{inputenc}
\usepackage[T1]{fontenc}
\usepackage{url}
%\usepackage{hyperref} javlja napako, če je odkomentirano
%%%%%%%%%%%%%%%%%%%%%%%%%%%%%%%%%%%%%%%%%%%%%%%%%%%%%%%%%%%%%%%%%%%%%%%%%%%%%
\def\B{\mathscr{B}^{1}} 
\def\R{\mathbb{R}}  % mnozica realnih stevil
{\theoremstyle{definition}
\newtheorem{zgled}[section]{Zgled}
}
%%%%%%%%%%%%%%%%%%%%%%%%%%%%%%%%%%%%%%%%%%%%%%%%%%%%%%%%%%%%%%%%%%%%%%%%%%%%%

% Thomaeova funkcija
% Beno Učakar

\begin{document}
\title{Thomaeova funkcija}
\author{Beno Učakar}
\date{29.~02.~2025}

Oglejmo si Thomaeovo funkcijo, ki je primer funkcije prvega Bairovega razreda.
Take funkcije lahko definiramo na naslednji način.
Naj bo \(D\subseteq \mathbb{R}\).

Funkcija \(f:D \to \mathbb{R} \) je \emph{funkcija prvega Bairovega razreda}, 
če obstaja funkcijsko zapor
edje $\{f_n\}$ zveznih funkcij na $D$, ki po točkah konvergira k $f$. 
Ta razred označimo z \(\B(D)\) oziroma, če ne bo nevarnosti zmede, kar z \(\B\).

% začetek zgleda
\begin{zgled}
Funkcijsko zaporedje ?? definiramo na sledeč način.
Za vsak p,q \(\in \mathbb{N}_0\), $1 \le q < n$ in $0 \le p \le q$ definiramo
\begin{itemize}
    \item \(f_n(x) = \max\left\{\frac{1}{n}, \frac{1}{q} + 2n^2\left(x - \frac{p}{q}\right)\right\}\) na intervalu \(\left(\frac{p}{q} - \frac{1}{2n^2}, \frac{p}{q}\right)\) in
    \item \(f_n(x) = \max\left\{\frac{1}{n}, \frac{1}{q} - 2n^2\left(x - \frac{p}{q}\right)\right\}\) na intervalu \(\left(\frac{p}{q}, \frac{p}{q} + \frac{1}{2n^2}\right)\).
\end{itemize}
V vseh ostalih točkah naj bo $f_n(x) = \frac{1}{n}$.
Preverimo lahko, da so intervali $\left(\frac{p}{q} - \frac{1}{2n^2}, \frac{p}{q} + \frac{1}{2n^2}\right)$ paroma disjunktni in zgornja definicija je dobra.
Opazimo, da je $f_n(x)$ odsekoma linearna zvezna. Če vzamemo limito po točkah, dobimo
\[
f(x) =
\begin{cases}
    \frac{1}{q}; \;\;\; x=\frac{p}{q} \; \text{je pokrajšan ulomek za \(p,\- q \in \mathbb{N} \) }
    \\
    0; \;\;\; x \; \text{je iracionalen}
\end{cases} 
\]
Pokazali smo, da \emph{Thomaeova funkcija} pripada \(\B\).
\end{zgled}
% konec zgleda

%%%%%%%%%%%%%%%%%%%%%%%%%%%%%%%%%%%%%%%%%%%%%%%%%%%%%%%%%%%%%%%%%%%%%
\bibliography{viri.bib}
\bibliographystyle{plain}
\nocite{*}
\end{document}