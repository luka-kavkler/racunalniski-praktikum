% \colon izpiše dvopičje v enačbi npr za predpis f: x -> y 


\documentclass[]{beamer}
\usepackage{amsmath}
\usepackage{amssymb}
% Naloga 1.3.1: Za dokument uporabite razred `beamer'.
% Ne dodajajte nastavitve za velikost pisave, kot je bila v datoteki `5-prosojnice.tex`.
\usepackage[T1]{fontenc}
\usepackage[utf8]{inputenc}
\usepackage[slovene]{babel}
% Naloga 1.3.2: vključite paket `predavanja'.
\usepackage{predavanja}
% Naloga 1.3.3: definirajte okolji `definicija' in `izrek'.
% Namig: z iskanjem po datotekah (Ctrl+Shift+F oz. Cmd+Shift+F) 
% poiščite niz `{definicija}' ali niz `{izrek}'.
{\theoremstyle{definition}
\newtheorem{definicija}{Definicija}
}
{\theoremstyle{plain}
\newtheorem{izrek}{Izrek}
}
\setbeameroption{hide notes}
\mode<presentation>
\begin{document}
% Naloga 1.3.4: pripravite naslovno stran z vsebino:
% - naslov: Matematični izrazi in uporaba paketa \texttt{beamer}
% - podnaslov: \emph{Matematičnih} nalog ni treba reševati!
% - inštitut: Fakulteta za matematiko in fiziko
% - datum: naj se ne izpiše; to dosežete z ukazom \date{}.
% Zgornje podatke nastavite z ukazi kot v dokumentih razreda `article`.
% Več o tem, kako se naredi naslovno stran, si preberite na naslovu na naslovu:
% https://www.overleaf.com/learn/latex/Beamer
% To stran preberite do vključno razdelka "Creating a table of contents".
% Ukaz `\titlepage` deluje podobno kot ukaz `\maketitle`, ki ste ga že srečali.
\title{Matematični izrazi in uporaba paketa \texttt{beamer}}
\subtitle{\emph{Matematičnih} nalog ni treba reševati!}
\institute{Fakulteta za matematiko in fiziko}
\date{}

\begin{frame}
    \titlepage
\end{frame}  


% Naloga 1.3.5: pripravite kazalo vsebine.
% 1. Naslov prosojnice, s kazalom vsebine naj bo "Kratek pregled"
% 2. S pomožnim parametrom `pausesections' (v oglatih oklepajih) 
%    določite, da naj se kazalo vsebine odkriva postopoma.
%    Poglejte, kako deluje ta ukaz.
% 3. Ker ni videti preveč lepo, pomožni parameter zakomentirajte.
\begin{frame}
    \frametitle{Kratek pregled}
    \tableofcontents%[pausesections]
\end{frame}
\section{Paket \texttt{beamer}}
%  Naslov prosojnice lahko naredimo tudi z dodatnim parametrom okolja `frame`.
\begin{frame}{Posebnosti prosojnic}
	% Naloga 2.3.1:
	% Dodajte ukaze, ki bodo poskrbeli, da se bo prosojnica odkrivala postopoma,
	% tako kot v datoteki prosojnice-resitev.pdf

	Za prosojnice je značilna uporaba okolja \texttt{frame},
	s katerim definiramo posamezno prosojnico, \pause
	%
	postopno odkrivanje prosojnic,\pause
	%
	ter nekateri drugi ukazi, ki jih najdemo v paketu \texttt{beamer}.\pause
	%
	\begin{exampleblock}{Primer}
		Verjetno ste že opazili, da za naslovno prosojnico niste uporabili
		ukaza \texttt{maketitle}, ampak ukaz \texttt{titlepage}.
	\end{exampleblock}
\end{frame}

\begin{frame}{Poudarjeni bloki}
	% Naloga 2.3.2:
	% Oblikujte poudarjena bloka z opombo in opozorilom.

		
		\begin{alertblock}{}
			Okolja za poudarjene bloke so \texttt{block}, \texttt{exampleblock} in \texttt{alertblock}.
		\end{alertblock}

		
		\begin{exampleblock}{}
			Začetek poudarjenega bloka (ukaz \texttt{begin}) vedno sprejme 
			dva parametra: okolje in naslov bloka.
			Drugi parameter (za naslov) je lahko prazen. 
		\end{exampleblock}

\end{frame}

\begin{frame}{Tudi v predstavitvah lahko pišemo izreke in dokaze}
	% Naloga 2.3.2:
	% Oblikujte okolje itemize, tako da se bo njegova vsebina postopoma odkrivala.
	% Ne smete uporabiti ukaza `pause'.
	% Beseda `največje' naj bo poudarjena šele na četrti podprosojnici.

	\begin{izrek}
	   Praštevil je neskončno mnogo.
	\end{izrek}
	\begin{proof}
	   Denimo, da je praštevil končno mnogo.
	   	% S pomožnim parametrom <+-> lahko določimo, da se bodo 
		% elementi naštevanja odkrivali postopoma.
	   \begin{itemize}[<+->]
		  \item Naj bo $p$ \alert<4>{največje} praštevilo.
		  \item Naj bo $q$ produkt števil $1$, $2$, \ldots, $p$.
		  \item Število $q+1$ ni deljivo z nobenim praštevilom, torej je $q+1$ praštevilo.
		  \item To je protislovje, saj je $q+1>p$. \qedhere
	   \end{itemize}
	\end{proof}
 \end{frame}
 
\section{Paketa \texttt{amsmath} in \texttt{amsfonts}}
\begin{frame}{Matrike}
	% Naloga 3.2.1:
	% Oblikujte determinanto matrike. 
	% Vsebina matrike je že pripravljena v komentarju spodaj.
	Izračunajte determinanto
\[		
	\begin{vmatrix}
		-1 & 4 & 4 & -2 \\
		 1 & 4 & 5 & -1 \\
		 1 & 4 & -2 & 2 \\
		 3 & 8 & 4 & 3 
	\end{vmatrix}
\]
	V pomoč naj vam bo Overleaf dokumentacija o matrikah:
	
	\href{https://www.overleaf.com/learn/latex/Matrices}{\beamergotobutton{Matrices}}
\end{frame}

\begin{frame}{Okolje \texttt{align} in \texttt{align*}}
	% Naloga 3.2.2:
	% Okolje align je namenjeno poravnavi enačb.
	% Če ne želimo, da se enačba oštevilči, uporabimo okolje align*.
	% Nadomestite prikazni matematični način z okoljem align*.
	% Na ustreznih mestih vključite & in \\, da bo enačba videti kot v rešitvi.
	% Za pravilno postopno odkrivanje morate na enem mestu uporabiti ukaz `only',
	% ter na dveh mestih ukaz `onslide'.
	Dokaži \emph{binomsko formulo}: za vsaki realni števili $a$ in $b$ in za vsako naravno število $n$ velja
	\begin{align*}
		(a+b)^n \only<1>{&=\ldots \\}
		\onslide<2-3>{&=(a+b)(a+b) \dots (a+b)\\}
		\onslide<3>{&= a^n + n a^{n-1} b + \dots + \binom{n}{k} a^{n-k} b^k + \dots + n a b^{n-1} + b^n}\\
		&= \sum_{k=0}^n \binom{n}{k} a^{n-k} b^k
	\end{align*}
	
\end{frame}

\begin{frame}{Še ena uporaba okolja \texttt{align*}}
	% Naloga 3.2.3:
	% Oblikujte spodnje enačbe z okoljem align*.
	% Če naredite po en kurzor na začetku vsake vrstice, 
	% boste lahko oblikovali vse tri vrstice hkrati.
	Nariši grafe funkcij:
	
	
	\begin{align*}
		&y = x^2 - 3|x| + 2  &  &y = 3 \sin(\pi+x) - 2 \\
		&y = \log_2(x-2) + 3 &  &y = 2 \sqrt{x^2+15} + 6 \\
		&y = 2^{x-3} + 1     &  &y = \cos(x-3) + \sin^2(x+1) \\
	\end{align*}
	
\end{frame}

\begin{frame}{Okolje \texttt{multline}}
	% Naloga 3.2.4:
	% Oblikujte spodnje enačbe z ustreznim okoljem,
	% da bo enačba oblikovana kot v rešitvah.
	Poišči vse rešitve enačbe
	
	
	\begin{align*}
		(1+x+x^2) \cdot (1+x+&x^2+x^3+\ldots+x^9+x^{10}) = \\
		&=(1+x+x^2+x^3+x^4+x^5+x^6)^2.
	\end{align*}
	
\end{frame}

\begin{frame}{Okolje \texttt{cases}}
	% Naloga 3.2.5:
	% Oblikujte spodnji funkcijski predpis z ustreznim okoljem,
	% da oblikovan kot v rešitvah.
	Dana je funkcija
	\[
		f(x,y)=
		\begin{cases}
			\frac{3x^2y-y^3}{x^2+y^2}; &(x,y)\neq(0,0)\\
			a; &(x,y)\neq(0,0)
		\end{cases}
	\]	
	\begin{itemize}
		% ukaz displaystyle preklopi v prikazni način v vrstici. 
	\item Določi $a$, tako da izračunaš limito \( \lim_{(x,y)\to(0,0)} f(x). \)
	\item Izračunaj parcialna odvoda $f_x(x,y)$ in $f_y(x,y)$.
	\end{itemize}
\end{frame}

\section[Matematika, 1. del\\\large{Analiza, logika, množice}]{Matematika, 1. del}
\begin{frame}{Logika in množice}
	\begin{enumerate}
		\item
		Poišči preneksno obliko formule 
		\(\exists x \, \colon P(x) \land \forall x \, \colon Q(x) \implies \forall x\, \colon R(x)\).
		\item 
		Definiramo množici \(A=[2,5]\) in \(B=\{0,1,2,3,4...\} \).
		V ravnino nariši:
		\begin{enumerate}
		   \item \(A \cap B \, \times \, \emptyset \)
		   \item \((A \cup B) \sim \Rightarrow \mapsto \int \,\)
		   %\dfrac rabis paket ams math
		\end{enumerate}
		\item
		Dokaži:
		\begin{itemize}
			\item ??
			\item ??
		\end{itemize}
	\end{enumerate}
\end{frame}

\begin{frame}{Analiza}
	\begin{enumerate}
		\item
		Pokaži, da je funkcija ?? enakomerno zvezna na ??.
		\item 
		Katero krivuljo določa sledeč parametričen zapis?
		% Spodaj si pomagajte z dokumentacijo o razmikih v matematičnem načinu.
		% https://www.overleaf.com/learn/latex/Spacing_in_math_mode
		$$
		   x(t) = a \cos t, ?? % tu manjka ukaz za presledek
		   y(t) = b \sin t, ?? % tu manjka ukaz za presledek
		   t \in [0, 2 \pi]
		$$ 
		\item
		Pokaži, da ima ?? inverzno funkcijo in izračunaj ??.
		
		\item
		Izračunaj integral 
		% V rešitvah smo spodnji integral zapisali v vrstičnem načinu,
		% ampak v prikaznem slogu. To naredite tako, da v matematičnem načinu najprej
		% uporabite ukaz displaystyle.
		% Pred dx je presledek: pravi ukaz je \,
		??
		% \frac{2+\sqrt{x+1}}{(x+1)^2-\sqrt{x+1}} 
		\item 
		Naj bo $g$ zvezna funkcija. Ali posplošeni integral 
		??
		konvergira ali divergira? Utemelji.
	\end{enumerate}
\end{frame}

\begin{frame}{Kompleksna števila}
	\begin{enumerate}
		\item
		Naj bo $z$ kompleksno število, $z \ne 1$ in ??.
		Dokaži, da je število \( i \, \frac{z+1}{z-1} \) realno.
		\item
		Poenostavi izraz:
		??
	\end{enumerate}
\end{frame}
\section{Stolpci in slike}

\section{Paket \texttt{beamer} in tabele}

\section[Matematika, 2. del\\\large{Zaporedja, algebra, grupe}]{Matematika, 2. del}

\end{document}